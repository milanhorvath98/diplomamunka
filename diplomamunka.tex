
%--------------------------------------------------------------------------------------
% Alapbeállítások
%--------------------------------------------------------------------------------------
\documentclass[12pt,a4paper,oneside]{report}
\usepackage[magyar]{babel} % Language support
\usepackage{geometry}
\usepackage{amsfonts,amsmath,amssymb} % Mathematical symbols
\usepackage{microtype} % Imrovements to typesetting
\usepackage{setspace} % For setting line spacing
\usepackage{cmap} % Enables more advenced text copying from the PDF document
\usepackage{sectsty} % Section heading styling

\usepackage{todonotes}

%\usepackage[backend=biber, sorting=none]{biblatex} % sorting=none a hivatkozások sorrendjéhez
%\addbibresource{mybib.bib} % A BibTeX adatbázisfájlod

\usepackage[unicode]{hyperref} % For hyperlinks in the generated document
\usepackage{booktabs} % For publication quality tables for LaTeX
\usepackage{graphicx} % For figure sizing
\usepackage{wrapfig}
\usepackage{float}
\usepackage[hang]{caption}
\usepackage{xcolor} % For code coloring in listings
\usepackage{listings} % For source code snippets
\usepackage[amsmath,thmmarks]{ntheorem} % Theorem-like environments

\usepackage[numbers]{natbib} % Bibliography

%\usepackage[T1]{fontenc} %Betűtípus beállításhoz

\usepackage{fontspec}
\setmainfont{Times New Roman}
\usepackage{lipsum}
\usepackage{tikz}
\usepackage{pdfpages}
%\usepackage{floatrow}

\newcommand{\tss}{\textsuperscript}     % tss = felső index


%--------------------------------------------------------------------------------------
% Elnevezések
%--------------------------------------------------------------------------------------
\newcommand{\elte}{Eötvös Loránd Tudományegyetem}
\newcommand{\ik}{Informatikai Kar}
\newcommand{\smi}{Savaria Műszaki Intézet}

\newcommand{\keszitette}{Horváth Milán}\newcommand{\konzulens}{Konzulens}
\newcommand{\temavezeto}{Témavezető}
\newcommand{\selectbsc}{
  \newcommand{\munkatipus}{Diplomamunka}       % Dokumentum típusa
  \newcommand{\munkatipusHU}{Diplomamunka}     % Dokumentum típusa
  \newcommand{\munkatipusok}{Diplomamunkák}   % többesszámban
  \newcommand{\munkatipustHU}{Diplomamunkát}  % tárgyraggal
}
\newcommand{\pelda}{Példa}
\newcommand{\definicio}{Definíció}
\newcommand{\tetel}{Tétel}
\newcommand{\jelolesek}{Jelölések jegyzéke}
\newcommand{\eloszo}{Előszó}
\newcommand{\bevezetes}{Bevezetés}
\newcommand{\koszonetnyilvanitas}{Köszönetnyilvánítás}
\newcommand{\osszefoglalas}{Összefoglalás}
\newcommand{\summary}{Summary}
\newcommand{\fuggelek}{Függelék}
\newcommand{\melleklet}{Mellékletek}
\newcommand{\authorName}{\authorFamilyName{} \authorGivenName}
\newcommand{\consulentA}{\consulentATitle\consulentAFamilyName{} \consulentAGivenName}
\newcommand{\consulentB}{\consulentBTitle\consulentBFamilyName{} \consulentBGivenName}
\newcommand{\consulentC}{\consulentCTitle\consulentCFamilyName{} \consulentCGivenName}
\newcommand{\supervisor}{\supervisorTitle\supervisorFamilyName{}
\supervisorGivenName}

\newcommand{\selectthesislanguage}{\selecthungarian}
\newcommand{\selectforeignlanguage}{\selectenglish}


\def\lstlistingname{lista}

\newcommand{\appendixletter}{6} % a fofejezet-szamlalo az angol ABC 6. betuje (F) lesz
\newcommand{\annexletter}{13} % M betű

%--------------------------------------------------------------------------------------
% Oldal elrendezés
%--------------------------------------------------------------------------------------
\pagestyle{plain}
\geometry{inner=35mm, outer=25mm, top=25mm, bottom=25mm}

%--------------------------------------------------------------------------------------
% Szöveg és bekezdés stílus
%--------------------------------------------------------------------------------------
\sectionfont{\Large\upshape\bfseries}  % Section title font
\subsectionfont{\Large\itshape\mdseries}
\subsubsectionfont{\large\itshape\mdseries}
\setcounter{secnumdepth}{3}             % Section numbering depth

\sloppy                                 % Prevent text from spilling over the margin
\widowpenalty=10000 \clubpenalty=10000  % Prevent widow and oprhan rows
\def\hyph{-\penalty0\hskip0pt\relax}    % Enable hyphenation

\onehalfspacing                         % 1.5x Line spacing

\newcommand{\selecthungarian}{
	\selectlanguage{magyar}
	\setlength{\parindent}{2em}			% Paragraph indentation
	\setlength{\parskip}{5pt}			% Paragraph spacing
	\frenchspacing
}

%--------------------------------------------------------------------------------------
% hyperref package beállítás
%--------------------------------------------------------------------------------------
\hypersetup{
    % bookmarks=true,            % show bookmarks bar?
    unicode=true,                % non-Latin characters in Acrobat's bookmarks
    pdfnewwindow=true,           % links in new window
    colorlinks=true,             % false: boxed links; true: colored links
    linkcolor=black,             % color of internal links
    citecolor=black,             % color of links to bibliography
    filecolor=black,             % color of file links
    urlcolor=black               % color of external links
}

%--------------------------------------------------------------------------------------
% Apply variables
%--------------------------------------------------------------------------------------
% This command is called in the main tex file and uses variables set there.
\newcommand{\applyvariables}{
	\author{\authorName}
	\title{\thesisTitle}

	\hypersetup{
		pdftitle={\thesisTitle},     % title
		pdfauthor={\authorName},     % author
		pdfsubject={\munkatipus}, % subject of the document
		pdfkeywords={\keywords},     % list of keywords (separate then by comma)
		pdfproducer={\authorName},   % producer of the document (organization)
		pdfcreator={LaTeX}           % creator of the document (application)
	}
}

%--------------------------------------------------------------------------------------
% Set up listings
%--------------------------------------------------------------------------------------
\definecolor{lightgray}{rgb}{0.95,0.95,0.95}
\lstset{
	basicstyle=\scriptsize\ttfamily, % print whole listing small
	keywordstyle=\color{black}\bfseries, % bold black keywords
	identifierstyle=, % nothing happens
	% default behavior: comments in italic, to change use
	% commentstyle=\color{green}, % for e.g. green comments
	stringstyle=\scriptsize,
	showstringspaces=false, % no special string spaces
	aboveskip=3pt,
	belowskip=3pt,
	backgroundcolor=\color{lightgray},
	columns=flexible,
	keepspaces=true,
	escapeinside={(*@}{@*)},
	captionpos=b,
	breaklines=true,
	frame=single,
	float=!ht,
	tabsize=2,
	literate=*
		{á}{{\'a}}1	{é}{{\'e}}1	{í}{{\'i}}1	{ó}{{\'o}}1	{ö}{{\"o}}1	{ő}{{\H{o}}}1	{ú}{{\'u}}1	{ü}{{\"u}}1	{ű}{{\H{u}}}1
		{Á}{{\'A}}1	{É}{{\'E}}1	{Í}{{\'I}}1	{Ó}{{\'O}}1	{Ö}{{\"O}}1	{Ő}{{\H{O}}}1	{Ú}{{\'U}}1	{Ü}{{\"U}}1	{Ű}{{\H{U}}}1
}

%--------------------------------------------------------------------------------------
% Setup captions
%--------------------------------------------------------------------------------------
\captionsetup[figure]{
	width=.75\textwidth,
	aboveskip=10pt}

\renewcommand{\captionlabelfont}{\it}
\renewcommand{\captionfont}{\footnotesize\it}



%%%%%%%%%%%%%%%%%%%%%%%%%%%%%%%%%%%%%%%%%%%%%%%%%%%%%%%%%%%%%%%%%%%%%%%%%%%%%%%%%%
%%%%%%%%%%%%%%%%%%%%%%%%%%%%%%%%%%%%%%%%%%%%%%%%%%%%%%%%%%%%%%%%%%%%%%%%%%%%%%%%%%
%%%%%%%%%%%%%%%%%%%%%%%%%%%%%%%%%%%%%%%%%%%%%%%%%%%%%%%%%%%%%%%%%%%%%%%%%%%%%%%%%%



\selectbsc
%--------------------------------------------------------------------------------------
% Változók beállítása [Setting variables]
%--------------------------------------------------------------------------------------
%TODO Állítsd be az alábbi változókat [Set these variables]%
% Szerző [Author]
\def\authorFamilyName{Horváth}
\def\authorGivenName{Milán}
\def\neptun{MYQGQ0}

% Konzulens 1 [Consulent 1]
\def\consulentATitle{}
\def\consulentAFamilyName{Őri}
\def\consulentAGivenName{Zsuzsanna}
\def\consulentARank{termékfejlesztő}

% Konzulens 2 [Consulent 2], ha nincs hagyd üresen
\def\consulentBTitle{}
\def\consulentBFamilyName{Kiss}
\def\consulentBGivenName{Henrik}
\def\consulentBRank{műszaki oktató}

% Konzulens 3 [Consulent 3], ha nincs hagyd üresen
\def\consulentCTitle{}
\def\consulentCFamilyName{}
\def\consulentCGivenName{}
\def\consulentCRank{}

% Témavezető
\def\supervisorTitle{}
\def\supervisorFamilyName{Bátorfi}
\def\supervisorGivenName{János György}
\def\supervisorRank{egyetemi tanársegéd}

% Dolgozat címe [Thesis title]
\def\thesisTitle{Diplomamunka}

% Kulcsszavak (a PDF-be) [Keywords (to PDF)]
\def\keywords{mechatronika, szabályozástechnika, etorobotika, ipar 4.0}

% Tanszék [Department]
\def\department{\smi}

% Elzártan kezelendő dolgozat [Restricted access]
%TODO Töltsd ki a korlátozás lejártának dátumát!
\def\endOfRestrictedAccess{2034 január 31. nap}

% Változók beállítása a PDF fájlhoz [Apply variables for the PDF file]
\applyvariables



%%%%%%%%%%%%%%%%%%%%%%%%%%%%%%%%%%%%%%%%%%%%%%%%%%%%%%%%%%%%%%%%%%%%%%%%%%%%%%%%%%
%%%%%%%%%%%%%%%%%%%%%%%%%%%%%%%%%%%%%%%%%%%%%%%%%%%%%%%%%%%%%%%%%%%%%%%%%%%%%%%%%%
%%%%%%%%%%%%%%%%%%%%%%%%%%%%%%%%%%%%%%%%%%%%%%%%%%%%%%%%%%%%%%%%%%%%%%%%%%%%%%%%%%



%--------------------------------------------------------------------------------------
% Dokumentum törzse [Document body]
%--------------------------------------------------------------------------------------
\onehalfspacing
\begin{document}

\pagenumbering{gobble}
\selectthesislanguage

% Címoldal [Titlepage]
\hypersetup{pageanchor=false}

%--------------------------------------------------------------------------------------
% Szennycímoldal [Cover title page]
%--------------------------------------------------------------------------------------

\clearpage
\begin{center}
\MakeUppercase{\authorName}\\[0.1cm]
\MakeUppercase{\munkatipus}\\[0.1cm]
\end{center}
\thispagestyle{empty}

%--------------------------------------------------------------------------------------
% Sorozatcímoldal [Series title page]
%--------------------------------------------------------------------------------------
\clearpage
\begin{center}

\MakeUppercase{\textbf{\elte}}\\[0.1cm]
\MakeUppercase{\textmd{\ik}}\\[0.1cm]
\MakeUppercase{\textmd{\department}}\\[0.8cm]
\vspace{1cm}
\includegraphics[width=40mm,keepaspectratio]{figures/elte_uj}\hspace{1cm}
\includegraphics[height=40mm,keepaspectratio]{figures/elte_ik_logo}\hspace{1cm}
\includegraphics[height=40mm,keepaspectratio]{figures/smi}\\[0.5cm]
\vspace{1cm}
\MakeUppercase{\munkatipusok}

\end{center}
\thispagestyle{empty}

%--------------------------------------------------------------------------------------
% Címoldal [Title page]
%--------------------------------------------------------------------------------------
\begin{titlepage}
\begin{center}

\MakeUppercase{\textbf{\elte}}\\[0.1cm]
\MakeUppercase{\textbf{\ik}}\\[0.1cm]
\MakeUppercase{\textbf{\department}}

\vspace{4.0cm}
{\huge \textsc{\authorName}}\\[0.8cm]
{\huge \MakeUppercase{\munkatipus}}\\[0.8cm]
{\Large \thesisTitle}

\vspace{3.0cm}

{
	\renewcommand{\arraystretch}{0.85}
	\begin{tabular}{ll}
	 \makebox[7cm][l]{\konzulens:} & \makebox[7cm][l]{\temavezeto:} \\
	 \noalign{\smallskip}
	 \makebox[7cm][l]{\hspace{1cm}\emph{\consulentA}} & \makebox[7cm][l]{\hspace{1cm}\emph{\supervisor}} \\
	 \makebox[7cm][l]{\hspace{1cm}\consulentARank} & \makebox[7cm][l]{\hspace{1cm}\supervisorRank} \\
	 \\
	 \makebox[7cm][l]{\hspace{1cm}\emph{\consulentB}} & \\
	 \makebox[7cm][l]{\hspace{1cm}\consulentBRank} & \\
	 \\
	 \makebox[7cm][l]{\hspace{1cm}\emph{\consulentC}} & \\
	 \makebox[7cm][l]{\hspace{1cm}\consulentCRank} & \\
	 
	\end{tabular}
}

\vfill
{\Large Szombathely, \the\year.}
\end{center}
\end{titlepage}
\hypersetup{pageanchor=false}
\thispagestyle{empty}
  % Diplomamunka címlap [Thesis]

% Copytightoldal [Copyright page]
%TODO Válaszd ki a megfelelőt! [Choose one]
%\include{copyrightpage}               % Nyílt kezelésű [Open access]
\selectlanguage{magyar}
\pagenumbering{gobble}
\selecthungarian
%--------------------------------------------------------------------------------------
% Copyrightoldal
%--------------------------------------------------------------------------------------
\begin{flushleft}
Szerzői jog {\textcopyright} \authorName, \the\year.
\end{flushleft}

\vspace{0.5cm}

\begin{center}
\textbf{ZÁRADÉK}\\
\end{center}

\vspace{0.5cm}
\noindent
Ez a \MakeLowercase{\munkatipusHU} elzártan kezelendő és őrzendő, a hozzáférése a vonatkozó szabályok szerint korlátozott, a diplomamunka tartalmát csak az arra feljogosított személyek ismerhetik.

A korlátozott hozzáférés időtartamának lejártáig az arra feljogosítottakon kívül csak a korlátozást kérelmező személy vagy gazdálkodó szervezet írásos engedélyéjével rendelkező személy nyerhet betekintést a diplomamunka tartalmába.

\vspace{0.3cm}

%TODO: fill out the date
A hozzáférés korlátozása és a zárt kezelés \endOfRestrictedAccess ján ér véget.

\vspace{30pt}
\noindent Szombathely, 2024. 01. 31.
\vfill
\clearpage
\thispagestyle{empty} % an empty page

\selectthesislanguage
   % Elzárt kezelésű [Restricted access]

% Feladatkiírás [Project page]
%TODO A nyomtatott verzóban ne szerepeljen! [Remove before printing]
%\include{chapters/project}

% Nyilatkozatok [Declarations]
\selectlanguage{magyar}
\selecthungarian
\pagenumbering{roman}
\setcounter{page}{6}
\cleardoublepage % duplexnél páratlan oldalon legyen
%--------------------------------------------------------------------------------------
% Nyilatkozatok
%--------------------------------------------------------------------------------------
\begin{center}
\section*{NYILATKOZATOK}
\end{center}

\vspace{0.5cm}


%--------------------------------------------------------------------------------------

\begin{center}
\emph{Nyilatkozat az önálló munkáról}
\end{center}
Alulírott,  \emph{\authorFamilyName{} \authorGivenName} (\neptun), az Eötvös Loránd Tudományegyetem hallgatója, büntetőjogi és fegyelmi felelősségem tudatában kijelentem és sajátkezű aláírásommal igazolom, hogy ezt a \MakeLowercase{\munkatipustHU} meg nem engedett segítség nélkül, saját magam készítettem, és szakdolgozatomban csak a megadott forrásokat használtam fel. Minden olyan részt, melyet szó szerint vagy azonos értelemben, de átfogalmazva más forrásból átvettem, egyértelműen, a hatályos előírásoknak megfelelően, a forrás megadásával megjelöltem.

Ennek a szakdolgozatnak önálló, eredeti szerzője vagyok, ez az önálló szellemi alkotás jogtisztaság szempontjából megfelel az „Eötvös Loránd Tudományegyetem Szervezeti és Működési Szabályzata, II. kötet, Hallgatói Követelményrendszer. Módosításokkal egybeszerkesztett változat [2017. szeptember 1.]” c. szabályzat 74/A–74/C. §-aiban foglalt rendelkezéseknek.

\begin{flushleft}
Szombathely, \today
\end{flushleft}

\begin{flushright}
 \makebox[7cm]{\rule{6cm}{.4pt}}\\
 \makebox[7cm]{\emph{hallgató}}
\end{flushright}


\vfill
\clearpage

\selectthesislanguage

\newcounter{romanPage}
\setcounter{romanPage}{\value{page}}
\stepcounter{romanPage}


\selectthesislanguage
% Tartalomjegyzék [Table of Contents]
\setcounter{tocdepth}{3}  % Tartalomjegyzék mélysége [ToC depth]
\tableofcontents\vfill

% Ábrák és táblázatok jegyzéke [List of Figures, Tables]
%TODO Kommenteld ki, ha használni szeretnéd. [Uncomment to use]
%\listoffigures\addcontentsline{toc}{chapter}{\listfigurename}   % Ábrák jegyzéke - opcionális
%\listoftables\addcontentsline{toc}{chapter}{\listtablename}     % Táblázatok jegyzéke - opcionális

\chapter*{Előszó}\addcontentsline{toc}{chapter}{\eloszo}
Már a középiskolás éveim során érdeklődtem a 3D tervezés, a CAD-CAM világa felé. Gépi forgácsoló szakmámból kifolyólag elég régóta kürölvesz engem a gépészeti világ és akkor jött a gondolat, mi lenne ha jelentkeznék egyetemre. Életem egyik legjobb döntése volt a gépészmérnöki képzés elkezdése. Rengeteg új információval gazdagodtam, sokkal jobban el tudtam mélyülni a CAD-CAM rendszerekben, valamint megismerkedtem számomra addig teljesen ismeretlen módszerekkel. Az egyik ilyen volt a végeselem analízis. Ez a terület tetszett meg a legjobban a képzés során, rengeteg lehetőség rejlik benne. A diplomamunka téma kiválasztásánál számomra fontos volt, hogy a CAD-CAM, valamint a végeselem analízis szerepet kapjanak az elkészítés során.


\begin{center}
    $\thicksim \; \thicksim \; \thicksim$
\end{center}


\subsubsection*{Köszönetnyilvánítás}
\emph{Elsőként szeretném megköszönni a TDK Hungary Components Kft.-nek, hogy a gépészmérnöki képzésem alatt biztosítottak számomra duális gyakorlati helyet, valamint hogy támogatták a diplomamunkám minőségi elkészültét. Szeretném megköszönni az Eurosolid Zrt.-nek, hogy biztosították számomra a Soldiworks 2022 Student Edition CAD szoftvert, amellyel a modelleket készítettem el.}

\vspace{0.5cm}

\begin{flushleft}
{Szombathely, \today}
\end{flushleft}

\begin{flushright}
\emph{\authorName}
\end{flushright}
\vfill

\chapter*{Jelölések}\addcontentsline{toc}{chapter}{\jelolesek}
%----------------------------------------------------------------------------

A táblázatban a többször előforduló jelölések magyar és angol nyelvű elnevezése,
valamint a fizikai mennyiségek esetén annak mértékegysége található. Az egyes
mennyiségek jelölése – ahol lehetséges – megegyezik hazai és a nemzetközi
szakirodalomban elfogadott jelölésekkel. A ritkán alkalmazott jelölések
magyarázata első előfordulási helyüknél található.

%~~~~~~~~~~~~~~~~~~~~~~~~~~~~~~~~~~~~~~~~~~~~~~~~~~~~~~~~~~~~~~~~~~~~~~~~~~~~~~~~~~~~~
% A táblázatokat ABC rendben kell feltölteni, először mindig a kisbetűvel
% kezdve. Ha egyazon betűjelnek több értelmezése is van, akkor mindegyiket kü-
% lön sorban kell feltüntetni. Konstansok esetén az értéket is a táblázatba
% kell írni.
% Dimenzió nélküli mennyiségek mértékegysége 1 és nem: – !
% A jelölésjegyzékben csak SI vagy SI-n kívüli engedélyezett mértékegységeket
% szabad feltüntetni. Egy dokumentumon belül az SI és pl. az angolszász
% mértékrendszer nem keverhető!
%~~~~~~~~~~~~~~~~~~~~~~~~~~~~~~~~~~~~~~~~~~~~~~~~~~~~~~~~~~~~~~~~~~~~~~~~~~~~~~~~~~~~~

%~~~~~~~~~~~~~~~~~~~~~~~~~~~~~~~~~~~~~~~~~~~~~~~~~~~~~~~~~~~~~~~~~~~~~~~~~~~~~~~~~~~~~
% A Jelölés oszlop alapvetően kurzív betűváltozattal szedendő, a Mértékegység
% oszlopot álló betűkkel kell szedni. Felső indexhez használható a \tss{}
% parancs.
%~~~~~~~~~~~~~~~~~~~~~~~~~~~~~~~~~~~~~~~~~~~~~~~~~~~~~~~~~~~~~~~~~~~~~~~~~~~~~~~~~~~~~

\def\arraystretch{1.5}%  vertical cell padding

\subsubsection*{Latin betűk}
\begin{center}
    \begin{tabular}{lp{10cm}l}
        \hline
        Jelölés & Megnevezés, megjegyzés, érték & Mértékegység \\
        \hline
        $E$     & Rugalmassági modulusz  & GPa     \\
        $F$     & erő                        & N           \\
        $S$     & keresztmetszet             & mm\tss{2} \\
        \hline
    \end{tabular}
\end{center}



\subsubsection*{Görög betűk}
\begin{center}
    \begin{tabular}{lp{10cm}l}
        \hline
        Jelölés & Megnevezés, megjegyzés, érték & Mértékegység \\
        \hline
                $\varepsilon$  & alakváltozás           & 1    \\
        $\sigma$  & feszültség                  & MPa             \\

        \hline
    \end{tabular}
\end{center}



\subsubsection*{Indexek, kitevők}
\begin{center}
    \begin{tabular}{lp{12.8cm}}
        \hline
        Jelölés & Megnevezés, értelmezés\\
        \hline
        $e$     & elem  \\
        max     & maximális érték        \\
        \hline
    \end{tabular}
\end{center}


\def\arraystretch{1}%  vertical cell padding

% Főszöveg [The main part of the thesis]
\cleardoublepage
\pagenumbering{arabic}
\chapter{Szakirodalmi áttekintés}

\section{Alapfogalmak és elméleti háttér}
A következőkben szeretnék kitérni a végeselem analízis témakörébe tartozó alapfogalmakra és a végeselem analízis elméleti hátterére. A fejezetben kitérek az alapvető elméletére, a merevségi mátrixra, a hálózásra, az anyagmodellekre, valamint a kezdeti- és peremfeltételekre.
\subsection{Alapvető elméleti koncepciók}
\textbf{Az energiaelv és a virtuális munka elve:}
\begin{itemize}
    \item \textbf{Az energiaelv:} Ezt az elvet arra használjuk, hogy megállapítsuk egy rendszer mechanikai állapotát, ahol az energia minimalizálása hozza létre az egyensúlyi állapotot. A potenciális energia minimális értéke az egyensúlyi konfigurációban fog megjelenni, ami alapvető a végeselem-módszerben történő modellezéskor.
    \item \textbf{A virtuális munka elve:} Egy másik alapvető koncepció, amely az anyagokon és szerkezeteken belüli belső erők és a külső terhelések közötti egyensúlyt írja le virtuális elmozdulásokon keresztül. Ezt az elvet arra használjuk, hogy a külső terhelések által végzett virtuális munka egyenlő legyen a belső erők által végzett virtuális munkával \cite{voros2012veges,pere2011vem}.
\end{itemize}
\newpage

\subsection{Merevségi mátrix}
A végeselem-analízis egyik kulcsfontosságú fogalma a merevségi mátrix, amely az egyes elemek merevségét reprezentálja és az egész szerkezet viselkedését határozza meg. Az egyes elemek merevségi mátrixait úgy tervezzük meg és aggregáljuk össze, hogy azok a globális szerkezeti modellt hozzák létre. Ez a globális mátrix tartalmazza az összes elemre vonatkozó merevségi információkat és lehetővé teszi, hogy a szerkezet teljes válaszát meghatározzuk a terhelések alatt.

A lokális merevségi mátrixok minden egyes végeselem számára külön-külön kerülnek meghatározásra, figyelembe véve az adott elem anyagi tulajdonságait, méreteit és a szerkezet geometriai viszonyait. Ezeket a lokális mátrixokat azután összeállítjuk egy globális merevségi mátrixba, amely a teljes szerkezet merevségi viszonyait írja le. Az aggregáció során figyelembe vesszük az elemek közötti csatlakozási pontokat és a határfeltételeket is.

A merevségi mátrix kialakítása a virtuális munka elvére és az energia megőrzésének elvére épül. Ez azt jelenti, hogy a szerkezet deformációjakor felhalmozódó belső rugalmas energia egyenlő a külső terhelések által végzett munkával. A virtuális munka elvének alkalmazásával a szerkezet egyensúlyi állapotát úgy írhatjuk le, hogy a virtuális elmozdulások esetén a belső és külső erők virtuális munkája nulla.

A merevségi mátrix fontos szerepet játszik abban, hogy modellezni tudjuk a szerkezeti elemek közötti kölcsönhatásokat. A mátrix elemei közvetlenül tükrözik az elemi csomópontok közötti merevségi kapcsolatokat, amelyek meghatározzák, hogyan oszlanak meg és terjednek el a belső erők és nyomatékok a szerkezeten belül. Ezáltal a merevségi mátrix nem csak az egyes elemek, hanem az elemek közötti kölcsönhatások viselkedését is leírja, lehetővé téve az analízis során a komplex terhelési állapotok pontos előrejelzését.

A merevségi mátrix létrehozása matematikailag a lokális elemi merevségi mátrixok egyesítésével történik, amelyeket a globális koordinátarendszerbe transzformálnak. Az alábbiakban egy egyszerűsített formában mutatom be a merevségi mátrix egy elemre vonatkozó képletét és a globális merevségi mátrixra való összeállítását. A merevségi mátrix $[K]$ az egyes elemek $D$ deformációs energiajának deriváltjaként jön létre a nodális elmozdulások u függvényében:
\begin{equation}
    [K] = \dfrac{\partial^2 D}{\partial u^2}
\end{equation}
ahol:
\begin{itemize}
    \item $[K]$: a merevségi mátrix,
    \item $D$: az elem deformációs energiája,
    \item $u$: a csomóponti elmozdulások vektora.
\end{itemize}

Egy konkrét elem $e$ merevségi mátrixát a következő képlet adja meg, ahol $E$ a rugalmassági modulus, $A$ az elem keresztmetszeti területe, $L$ az elem hossza, és az integrálást az elem hossza mentén végzik:
\begin{equation}
    [K^{(e)}] = \frac{E A}{L}
    \begin{bmatrix}
        1   &   -1 \\
        -1  &   1
    \end{bmatrix}
\end{equation}

Az egész szerkezetre vonatkozó globális merevségi mátrix a következőképpen áll össze az egyes elemek lokális merevségi mátrixainak összeállításával és a megfelelő csomópontokhoz való hozzárendeléssel:

\begin{equation}
    [K_{\text{global}}] = \sum_{e=1}^{n} [T^{(e)}]^\top [K^{(e)}] [T^{(e)}]
\end{equation}
itt:
\begin{itemize}
    \item $[K^{(e)}]$: az $e$-edik elem lokális merevségi mátrixa,
    \item $[T^{(e)}]$: a transzformációs mátrix, ami az $e$-edik elem lokális koordinátáit transzformálja a globális koordinátarendszerbe,
    \item $n$ az elemek száma a szerkezetben.
\end{itemize}

A transzformációs mátrixok és a globális koordinátarendszerbe történő integrálás biztosítja, hogy minden elem merevségi jellemzői megfelelően szerepeljenek a teljes szerkezet merevségi viselkedésének modellezésében \citep{voros2012veges,paczelt2007veges,VEM2011}.

\subsection{Hálózás}
A hálózás a végeselem-analízis (FEM) egyik alappillére, hiszen a komplex geometriai formák és szerkezeti problémák számítását a véges elemekre való felosztás teszi lehetővé. A háló esszenciálisan egy digitális rács, ami a fizikai testet kisebb, kezelhető elemekre, úgynevezett véges elemekre bontja le. Ezen elemek közötti kapcsolatok és kölcsönhatások modellezik a teljes szerkezet viselkedését a valóságban jelentkező terhelések és környezeti hatások alatt.

A hálózás során az a cél, hogy a modell geometriáját és a szerkezeti viselkedését lehető legpontosabban leképezzük, miközben gazdaságosak maradunk a számítási erőforrásokkal. A háló kialakítása során olyan döntéseket kell meghozni, amelyek befolyásolják az analízis pontosságát és hatékonyságát, mint például az elemek mérete, alakja és eloszlása. Az ideális háló finoman követi a geometriai kontúrokat, elkerülve a túl nagy vagy túl kis elemeket, amelyek torzíthatják az eredményeket vagy feleslegesen növelik a számítási igényeket.

A háló minőségének biztosítása elengedhetetlen a megbízható FEM eredményekhez. Ha a háló nem megfelelően van felosztva, pontatlan eredményeket adhat, és ezen felül számítási hibákhoz is vezethet. Ebből adódóan, a hálózás nem egyszerűen a modell részekre történő bontása, hanem kritikus lépés a szerkezeti elemzésben, amely megalapozza a további analízis pontosságát és érvényességét.

A lineáris és kvadratikus elemek kiválasztása során figyelembe kell venni a vizsgálati célt, a rendelkezésre álló számítási erőforrásokat, valamint a kívánt pontosság és felbontás egyensúlyát. A kvadratikus elemek alkalmazása ajánlott ott, ahol az eredmények nagy pontossága elengedhetetlen, különösen ott, ahol a szerkezeti viselkedésben nemlineáris vagy nagyfokú geometriai változások vannak. Az ilyen elemek használata növelheti a számítási időt, de csökkentheti a szükséges elemek összmennyiségét a hálóban, optimalizálva ezzel a számítások gazdaságosságát. A lineáris elemek egyszerűbb esetekben, ahol a pontosság kevésbé kritikus, gyors és hatékony megoldást nyújthatnak. Az adaptív hálózati stratégiák segítségével tovább finomítható a háló a szükséges pontosság és számítási hatékonyság elérése érdekében \cite{tamas2014vegeselem}.

\subsection{Anyagtulajdonságok és anyagmodellek}
Az anyagtulajdonságok megfelelő meghatározása létfontosságú a végeselem-analízisben, hiszen ezek a paraméterek határozzák meg, hogy a szimulált szerkezeti elemek hogyan reagálnak a terhelésekre. A szakítópofák esetében különösen fontos, hogy az anyagtulajdonságok pontosan tükrözzék a valós viselkedést, mivel ezek a szerkezeti komponensek gyakran kerülnek maximális terhelés alá a vizsgálatok során.

\textbf{Rugalmassági modulus $(E)$:} Az anyag merevségét jellemzi, és azt fejezi ki, hogy az anyag mennyire ellenáll az alakváltozásnak. Nagyobb modulus esetén az anyag kevésbé deformálódik azonos feszültség hatására.

\textbf{Poisson-tényező $(\nu)$:} Ez az arányossági tényező leírja az anyag hosszanti megnyúlása és a keresztirányú összehúzódása közötti viszonyt.

\textbf{Szakítószilárdság $(\sigma_u)$:} Az a maximális feszültség, amit az anyag elvisel törés nélkül. Ez egy kritikus paraméter a törési analízisben.

Az anyagmodell kiválasztása a vizsgálandó jelenségektől függ. Ha az anyag viselkedése a vizsgált terhelési tartományban lineárisan rugalmas, akkor egy lineáris rugalmas modell alkalmazása elegendő. Itt az anyag viselkedése a Hooke-törvény által leírt módon lineáris, azaz az anyagban ébredő feszültség arányos az alakváltozással. Ebben az esetben az anyagmodell csak a rugalmassági modulust és a Poisson-tényezőt foglalja magában.

Elasztikus-plasztikus modellek alkalmazása akkor szükséges, ha az anyag átlépi a rugalmas tartományát és plasztikus alakváltozást szenved. Ezen modellek paramétereit sokkal összetettebb kalibrálni, mert figyelembe kell venni a folyáshatárt, a keményedési tulajdonságokat és a törési viselkedést is.

A modellek paramétereinek kalibrálása során kísérleti adatokra vagy megbízható szakirodalmi forrásokra támaszkodunk. A kalibrálás azt jelenti, hogy a szimulációs modellben használt paramétereket úgy állítjuk be, hogy azok a valós anyagviselkedést a lehető legjobban közelítsék. A validálás azt a folyamatot jelenti, hogy összehasonlítjuk a szimulációs eredményeket kísérleti adatokkal vagy más megbízható forrásból származó eredménnyel, ezzel bizonyítva a modell megbízhatóságát és pontosságát. Egy jól kalibrált és validált anyagmodell növeli a végeselem-analízis eredményeinek hitelességét, így biztosítva, hogy a szakítópofa tervezésekor kapott eredmények megbízhatóak legyenek \cite{tamas2014vegeselem}.

\subsection{Kezdeti- és peremfeltételek}
A perem- és kezdőfeltételek meghatározása kritikus lépés a végeselem-modellezésben, mivel ezek határozzák meg, hogy a modell hogyan viselkedik a terhelés alatt és milyen korlátok között mozoghat. Ezeket a feltételeket úgy választjuk meg, hogy azok pontosan tükrözzék a valós körülményeket.

\noindent\textbf{Peremfeltételek: }
A terhelési pontok meghatározása elengedhetetlen, mert ezek határozzák meg, hogy milyen erők és nyomatékok hatnak a szerkezetre. A rögzítési pontok vagy felületek megadása szintén fontos, mert ezek jelölik ki, hogy a szerkezet mely pontjainál van megakadályozva az elmozdulás, így modellezve a valós rögzítési viszonyokat. A peremfeltételek beállítása során ügyelni kell arra, hogy azok ne legyenek túl merevek vagy túl engedékenyek, mivel ez befolyásolhatja a szerkezeti válaszokat és az eredmények pontosságát.

\noindent\textbf{Kezdőfeltételek: }
Ezek a feltételek a vizsgált szerkezeti elem kezdeti állapotát írják le, mint például előfeszítést vagy kezdeti elmozdulásokat. A kezdőfeltételek meghatározásával a modellben szimulálhatjuk azokat a kezdeti viszonyokat, amelyek a terhelés megkezdése előtt állnak fenn \cite{mankovits2015modellezes}.

\noindent\textbf{Dinamikai és nemlineáris hatások: }
A valósághű modellezéshez fontos a dinamikai hatások, mint tömeg és inercia által okozott erők figyelembe vétele, különösen gyors terhelések vagy rezgések esetén. A nemlineáris jelenségek, mint nagy elmozdulások vagy anyagviselkedés, szintén lényegesek, különösen a rugalmassági határon túli terhelések esetén. Ezek a jellemzők jelentősen befolyásolják a szerkezet válaszát és az eredmények pontosságát, ezért fontosak a megbízható elemzéshez.

\chapter{Probléma}

\chapter{Tervezés}

\chapter{Végeselem analízis}

\chapter{Gyártás}

\chapter{Összegzés}



\footnotesize  % Kisebb betűméret [Smaller font size]
\bibliographystyle{unsrtnat}
\bibliography{bib/mybib}
\newpage
%\printbibliography

% Függelék és mellékletek [Appendices]
\appendix
\newgeometry{inner=20mm, top=10mm, bottom=20mm, outer=25mm}
% Melléklet A
\begin{center}
    \Large\textbf{Melléklet A}
\end{center}
%\includepdf[pages=1, scale=0.9]{Szakítópofa_új.pdf}
\newpage

% Melléklet B
\begin{center}
    \Large\textbf{Melléklet B}
\end{center}
%\excludeFromLocAndLot % A következő ábrákat és a táblázatokat hagyja ki a jegyzékből
                      % [Exclude following figures and tables from List Of Figures/Tables]


\end{document}
