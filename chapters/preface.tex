%----------------------------------------------------------------------------
\chapter*{\eloszo}\addcontentsline{toc}{chapter}{\eloszo}
%----------------------------------------------------------------------------

Az előszó legtöbbször személyes hangú, eligazító jellegű írás, amely a mű megírásának okairól, születésének körülményeiről szól. Az előszó nem szerves része a főszövegnek, hanem annak kiegészítése.
Ugyancsak az előszóban fejtheti ki a szerző a mű megértéséhez szükséges szempontokat, a követett módszereket, utalhat a fontosabb előzményekre és szakirodalomra.
Az előszó ne legyen terjedelmes.


\begin{center}
    $\thicksim \; \thicksim \; \thicksim$
\end{center}


\subsubsection*{Köszönetnyilvánítás}
\emph{A köszönetnyilvánítás ide írható.}

\vspace{0.5cm}

\begin{flushleft}
{Szombathely, \today}
\end{flushleft}

\begin{flushright}
\emph{\authorName}
\end{flushright}

\vfill
